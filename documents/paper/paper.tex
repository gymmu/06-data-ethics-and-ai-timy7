\documentclass{report}

\usepackage[ngerman]{babel}
\usepackage[utf8]{inputenc}
\usepackage[T1]{fontenc}
\usepackage{hyperref}
\usepackage{csquotes}
\usepackage[a4paper]{geometry}

\usepackage[
    backend=biber,
    style=apa,
    sortlocale=de_DE,
    natbib=true,
    url=false,
    doi=false,
    sortcites=true,
    sorting=nyt,
    isbn=false,
    hyperref=true,
    backref=false,
    giveninits=false,
    eprint=false]{biblatex}
\addbibresource{../references/bibliography.bib}


\title{Ethik im Umgang mit Daten}
\author{Yeva Skotar}
\date{\today}


\begin{document}

\maketitle

\abstract{
    Dies ist eine Vorlage für eine Maturarbeit in der Informatik am Gymnasium Muttenz. Sie dient dazu, die Arbeit schnell und einfach zu starten und sollte einen guten Überblick über die Arbeit bieten.
}

\tableofcontents

\chapter{Einleitung}
Unsere Welt ist mit verschiedenen Daten überquert. Die Informationsquellen können eine Enzeklopedie, ein Menu im Restaurant, Werbung auf den Strassen oder Google sein.
Täglich sind wir am Daten Suchen, und es ist uns fast unmöglich die Rescherschen zu beenden, weil wir immer etwas neues, bisher noch unbekanntes 
begegnen und ein wenig Informationen darüber zu bekommen wollen.
Deswegen sind die Daten ein wichtiger Bestandteil für uns in fast allen Bereichen von unseren Leben 
und wir streben nach Möglichkeit die uns notwendige Daten so schnell wie möglich zu bekommen 
und aber die Qalität von den Informationen soll immernoch hoch sein. So haden die gescheidene Menschen Künstliche Inteligenz
entwikelt und damit kann die Menschheit in wenigen Sekunden allerleie Fragen beantworten haben ohne Artickel darüber zu lesen.
Toll oder? Aber was wenn die Menschheit wird sich so sehr auf KI verlassen 
dass es zu einzige Datenquelle wird? Die Daten sind nicht nur das Wissen aber auch ein entscheidener Faktor von unserem Verhalten.
Je grösser die Auswahl von Optionen von Verhalten wir kennen desto schwieriger fehl uns der Entschied aber je weniger sind es desto begrenzter sind unsere Handlungen 
und 

Hier kommt die Einführung. Der Text hier sollte eigentlich noch viel länger sein, so das hier nicht so merkwürdige Umbrüche entstehen.

Ich kann weitere Kapitel auch importieren.

\input{chap_methode.tex}

\section{Etwas mit Quellen}

Etwas mit Änderung hier am Ende.

Wenn ich eine Quelle zitieren möchte, kann ich das ganze einfach am Ende des Satzes machen \citep{example}. Oder wie \citet{example} sagt, auch mitten im Text.

\printbibliography

\end{document}
