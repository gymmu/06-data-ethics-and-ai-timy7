\documentclass{report}

\usepackage[ngerman]{babel}
\usepackage[utf8]{inputenc}
\usepackage[T1]{fontenc}
\usepackage{hyperref}
\usepackage{csquotes}
\usepackage[a4paper]{geometry}

\usepackage[
    backend=biber,
    style=apa,
    sortlocale=de_DE,
    natbib=true,
    url=false,
    doi=false,
    sortcites=true,
    sorting=nyt,
    isbn=false,
    hyperref=true,
    backref=false,
    giveninits=false,
    eprint=false]{biblatex}
\addbibresource{../references/bibliography.bib}


\title{Ethik im Umgang mit Daten}
\author{Yeva Skotar}
\date{\today}


\begin{document}

\maketitle

\abstract{
    Dies ist eine Vorlage für eine Maturarbeit in der Informatik am Gymnasium Muttenz. Sie dient dazu, die Arbeit schnell und einfach zu starten und sollte einen guten Überblick über die Arbeit bieten.
}

\tableofcontents

\chapter{Einleitung}
Unsere Welt ist mit verschiedenen Daten überquert. Die Informationsquellen können eine Enzeklopedie, ein Menu im Restaurant, Werbung auf den Strassen oder Google sein.
Täglich sind wir am Daten Suchen, und es ist uns fast unmöglich die Rescherschen zu beenden, weil wir immer etwas neues, bisher noch unbekanntes 
begegnen und ein wenig Informationen darüber zu bekommen wollen.
Deswegen sind die Daten ein wichtiger Bestandteil für uns in fast allen Bereichen von unseren Leben 
und wir streben nach Möglichkeit die uns notwendige Daten so schnell wie möglich zu bekommen 
und aber die Qalität von den Informationen soll immernoch hoch sein. So haben die gescheidene Menschen Künstliche Inteligenz
entwikelt und damit kann die Menschheit in wenigen Sekunden allerleie Fragen beantworten haben, ohne auf ihre Fragestellung zu achten im Gegensatz zu Suchmaschienen.
Toll oder? Aber was wenn die Menschheit wird sich so sehr sich auf KI verlassen 
dass es zu einzige Datenquelle wird, wegen ihrer einfacher Benützungsweisse? Mit der jetziger Tendenz KI zu verwenden ist es nicht ausgeschlossen. ChatGPT und andere KI-tools ähneln sich auf
 einen Gesprächspartner bei der Beantwortung auf die Fragen, deswegen heissen sie Künstliche Inteligenz - tools,aber ob die Informationen stimmen können sie nicht nachweisen und haben keine Verantwortung dafür. 
Es ist den Menschen überlassen 
ob sie die Daten überprüfen oder nicht. Wie viel Prozent von Leuten nach der Anfrage bei KI, dann zusätzlich noch recherschieren ob es um die aktuelle, wahre Informationen handelt?
 Daten sind nicht nur das Wissen aber auch ein entscheidener Faktor von unserem Verhalten während wir Entscheidungen treffen müssen.
Je mehr Informationen wir verfügen desto mehr Möglichkeiten für die Lösung unserer Problemen haben wir und es bezieht sich praktisch auf alle Lebensbereichen,
 uns fehlt der Entschied zwar manchmal schwieriger als wenn die Auswahl von unseren Möglichkeiten reduziert ist aber je weniger wissen wir desto begrenzter sind unsere Handlungen.
Das kann dazu führen das zwischen uns bekannten Möglichkeiten uns gar keine passt. 

\subsection{Inhaltsveryeichnis}

\begin{itemize}
\item[-] Was ist KI und welche Daten besitzt die sie?
\item[-] Unterschieden zwischen ChatGPT und Google
\item[-] Statistiken zu KI-tools Benützungsheufigkeit und Qualität
\item[-]  
\item[-] Wird KI die Menschen manipulieren können?

\end{itemize}

\chapter{Was ist KI und welche Daten besitzt die sie?}

Künstliche Intelligenz ist die Fähigkeit einer Maschine, menschliche Fähigkeiten wie logisches Denken, 
Lernen, Planen und Kreativität zu imitieren. Die Imitation ist nicht nur in den Fähigkeiten sondern auch in den 
Funktionsweisse von KI ihre Aufbau. Analog zu menschlichen Gehirn besitzt die KI künstlichen neuronallen Netzen und 
deren Verknüpfungen Synapsen.



 Generell wird das intelligente Verhalten der 
Technologien mit Mitteln der Informatik und Mathematik/Statistik simuliert.
 Die Computer werden für bestimmte Aufgaben trainiert, häufig indem sie große Datenmengen 
 verarbeiten und darin Muster erkennen. Das heisst KI Trainning, sie gibt der KI Fähichkeiten damit sie überhaupt funktionieren kann. 

 \section{Deep Learning} 
  Deep Learning ist eine Methode des KI Trainning,
  bei der ein Algorithmus mithilfe von künstlichen neuronalen Netzen lernt, Zusammenhänge in besonders
  großen Datenmengen zu erkennen bzw. abzubilden.








\input{chap_methode.tex}

\section{Etwas mit Quellen}

Etwas mit Änderung hier am Ende.

Wenn ich eine Quelle zitieren möchte, kann ich das ganze einfach am Ende des Satzes machen \citep{example}. Oder wie \citet{example} sagt, auch mitten im Text.


\begin{itemize}
\item \url{https://www.produktion.de/technik/zukunftstechnologien/kuenstliche-intelligenz/kuenstliche-intelligenz-verstaendlich-erklaert-243.html#wie-funktioniert-kuenstliche-intelligenz}
\item \url{https://www.europarl.europa.eu/topics/de/article/20200827STO85804/was-ist-kunstliche-intelligenz-und-wie-wird-sie-genutzt}
\item \url{https://digitalzentrum-augsburg.de/kuenstliche-intelligenz-einfach-erklaert/}
\end{itemize}

\printbibliography

\end{document}
