\documentclass{article}

\usepackage[ngerman]{babel}
\usepackage[utf8]{inputenc}
\usepackage[T1]{fontenc}
\usepackage{hyperref}
\usepackage{csquotes}

\usepackage[
    backend=biber,
    style=apa,
    sortlocale=de_DE,
    natbib=true,
    url=false,
    doi=false,
    sortcites=true,
    sorting=nyt,
    isbn=false,
    hyperref=true,
    backref=false,
    giveninits=false,
    eprint=false]{biblatex}
\addbibresource{../references/bibliography.bib}

\title{Notizen zum Projekt Data Ethics}
\author{Yeva Skotar}
\date{\today}

\begin{document}
\maketitle

\abstract{
    Dieses Dokument ist eine Sammlung von Notizen zu dem Projekt. Die Struktur innerhalb des
    Projektes ist gleich ausgelegt wie in der Hauptarbeit, somit kann hier einfach geschrieben
    werden, und die Teile die man verwenden möchte, kann man direkt in die Hauptdatei ziehen.
}

\tableofcontents

\input{section_ai.tex}

Welche Themen sind relevant/ Fragestellung
-- welche dateien Besitzt die KI und was kann sie damit machen  
-- was sind die Vorteile und Nachteie von Daten aus KI 
-- wie wird die KI verwendet und was könnte darin gefährlich sein 
--  


Welche Daten Besitzt die KI 
Internet, alle mögliche Sensoren und öffentliche Datensätze sind Quelen von allerleien Informationen 
für die KI 



\printbibliography

\end{document}
