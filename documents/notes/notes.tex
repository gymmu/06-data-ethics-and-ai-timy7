\documentclass{article}

\usepackage[ngerman]{babel}
\usepackage[utf8]{inputenc}
\usepackage[T1]{fontenc}
\usepackage{hyperref}
\usepackage{csquotes}

\usepackage[
    backend=biber,
    style=apa,
    sortlocale=de_DE,
    natbib=true,
    url=false,
    doi=false,
    sortcites=true,
    sorting=nyt,
    isbn=false,
    hyperref=true,
    backref=false,
    giveninits=false,
    eprint=false]{biblatex}
\addbibresource{../references/bibliography.bib}

\title{Notizen zum Projekt Data Ethics}
\author{Yeva Skotar}
\date{\today}

\begin{document}
\maketitle

\abstract{
    Dieses Dokument ist eine Sammlung von Notizen zu dem Projekt. Die Struktur innerhalb des
    Projektes ist gleich ausgelegt wie in der Hauptarbeit, somit kann hier einfach geschrieben
    werden, und die Teile die man verwenden möchte, kann man direkt in die Hauptdatei ziehen.
}

\tableofcontents

\input{section_ai.tex}

Plan:
so viel wie möglich Themen die ich schon vorhabe zu bearbeiten -- alles aus den Quellen finden und aufschreiben, 
nicht vergessen Comits zu machen 
Unterricht:
Bild einfügen, die Arbeit im PDF öffnen und den Aussehen verändern 

Themen zu Ethik:
-- wieso entwikelt man die KI tools stendig und den menschlichen Gechirn oft nur in den Schulen?
-- die KI erleichtert den Menschen Lösung ihrer Problemen zu finden, werden wir im Entwicklungsprozess der KI 
selber auch entwikeln oder der KI nur folgen.
-- ist Künstliche Inteligenz inteligenter als Mensch?
-- wie sehr zuverlässtman sich auf die KI

Welche Themen sind relevant/ Fragestellung
-- welche dateien Besitzt die KI
-- wie kann sich die Künstliche Inteligenz entwickeln/ Daten Aktualisieren und sich auf dem Benützer anpassen 
-- was untrerscheidet Anfrage bei der KI und Anfrage im Browser 
-- was sind die Vorteile und Nachteie von Daten aus KI 
-- wie wird die KI verwendet
--  

-- werden Menschen weniger vielfalt an Daten mit KI bekommen als bei Einfacher Internetsuche 

Einleitung: Funktionsweise von der KI 


eine Künstliche Inteligenz 

Welche Daten Besitzt die KI 
Internet, alle mögliche Sensoren und öffentliche Datensätze sind Quelen von allerleien Informationen 
für die KI 

Unsere Welt ist mit verschiedenen Daten überquert. Die Informationsquellen können eine Enzeklopedie, ein Menu im Restaurant, Werbung auf den Strassen oder Google sein.
Täglich sind wir am Daten Suchen, und es ist uns fast unmöglich die Rescherschen zu beenden, weil wir immer etwas neues, bisher noch unbekanntes 
begegnen und ein wenig Informationen darüber zu bekommen wollen.
Deswegen sind die Daten ein wichtiger Bestandteil für uns in fast allen Bereichen von unseren Leben 
und wir streben nach Möglichkeit die uns notwendige Daten so schnell wie möglich zu bekommen 
und aber die Qalität von den Informationen soll immernoch hoch sein. So haben die gescheidene Menschen Künstliche Inteligenz
entwikelt und damit kann die Menschheit in wenigen Sekunden allerleie Fragen beantworten haben, ohne auf ihre Fragestellung zu achten im Gegensatz zu Suchmaschienen.
Toll oder? Aber was wenn die Menschheit wird sich so sehr sich auf KI verlassen 
dass es zu einzige Datenquelle wird, wegen ihrer einfacher Benützungsweisse? Mit der jetziger Tendenz KI zu verwenden ist es nicht ausgeschlossen. ChatGPT und andere KI-tools ähneln sich auf einen Gesprächspartner bei der Beantwortung auf die Fragen, aber ob die Informationen stimmen können sie nicht nachweisen und haben keine Verantwortung dafür. 
Es ist den Menschen überlassen 
ob sie die Daten überprüfen oder nicht. Wie viel Prozent von Leuten nach der Anfrage bei KI, dann zusätzlich noch recherschieren ob es um die aktuelle, wahre Informationen handelt? Daten sind nicht nur das Wissen aber auch ein entscheidener Faktor von unserem Verhalten während wir Entscheidungen treffen müssen.
Je mehr Informationen wir verfügen desto mehr Möglichkeiten für die Lösung unserer Problemen haben wir und es bezieht sich praktisch auf alle Lebensbereichen, uns fehlt der Entschied zwar manchmal schwieriger als wenn die Auswahl von unseren Möglichkeiten reduziert ist aber je weniger wissen wir desto begrenzter sind unsere Handlungen.
Das kann dazu führen das zwischen uns bekannten Möglichkeiten uns gar keine passt. 


\printbibliography

\end{document}
