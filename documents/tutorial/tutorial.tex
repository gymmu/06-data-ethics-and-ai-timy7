\documentclass{article}

\usepackage[ngerman]{babel} %Sprachelemente wie z.B. Silbentrennung
\usepackage[utf8]{inputenc} %Damit Umlaute und Sonderzeichen direkt eingegeben werden können
\usepackage[T1]{fontenc}
\usepackage{hyperref}
\usepackage{csquotes}
\usepackage[a4paper]{geometry}
\usepackage{xcolor} %Farben
\usepackage{setspace} %Zeilenabstand definieren
\usepackage{graphicx} %Um externe Bilddateien einfügen zu können.
\usepackage{hyperref} %Verlinkungen
\usepackage{multicol} %Mehrere Spalten
\usepackage{float}
\usepackage{caption}

\usepackage[
    backend=biber,
    style=apa,
    sortlocale=de_DE,
    natbib=true,
    url=false,
    doi=false,
    sortcites=true,
    sorting=nyt,
    isbn=false,
    hyperref=true,
    backref=false,
    giveninits=false,
    eprint=false]{biblatex}
    \addbibresource{../references/bibliography.bib}
    
%\renewcommand*\familydefault{\sfdefault} %Falls keine Serifen gewünscht

% Definiere neue Theoreme. Diese können als Baublöcke für Aufgaben oder ähnliches verwendet werden.
\newtheorem{theorem}{Theorem}[section]
\newtheorem{rem}[theorem]{Remark}
\newtheorem{ex}[theorem]{Übung}

% Setze den Zeilenabstand
\onehalfspacing %1.5 facher Zeilenabstand

% Definiere automatische Übersetzungen.
\addto{\captionsngerman}{
	\renewcommand\refname{Literaturverzeichnis} %Name für Quellenverzeichnis
}

\title{Eine kurze Einführung in \LaTeX}
\author{C. Geissmann}

% Hier wird das Dokument gestartet. Alles was darin ist, wird im Dokument abgedruckt.
\begin{document}

% Hier wird die Titelseite gestaltet.
\begin{titlepage}
    \makeatletter % Das hier brauchen wir damit wir spezielle Befehle wie \@author verwenden können.
	\begin{center}
		{\scshape Gymnasium Muttenz} \vspace{0.5cm}

		 Informatik 2023/2024\vspace{5.5cm}

		{\huge\bfseries \@title}

		\vspace{2cm}

		{\Large\itshape \@author}

        \vspace{2cm}

        Version vom: \@date
	\end{center}
    
    \makeatother % Wir müssen das @ wieder schliessen, damit der Rest ganz normal funktioniert.
\end{titlepage}
\newpage

\setcounter{page}{1} %Seitennummern nach Titelseite mit 1 beginnen
\tableofcontents %Inhaltsverzeichnis einblenden
\clearpage %Seitenumbruch


\section{Einleitung}
\subsection{\TeX und \LaTeX}

\TeX\, (sprich „Tech“, kann auch „TeX“ geschrieben werden) ist ein Computerprogamm von Donald E. Knuth.
Es dient zum Setzen von Texten und mathematischen Formeln.
\vspace{2mm}

% Leerzeilen sorgen für einen Abschnittsumbruch

\LaTeX\, (sprich „Lah-tech“ oder „Lej-tech“, kann auch „LaTeX“ geschrieben werden) ist ein auf \TeX\, aufbauendes Computerprogramm und wurde von Leslie Lamport geschrieben. Es vereinfacht den Umgang mit  \TeX, indem es entsprechend der logischen Struktur des Dokuments auf vorgefertigte Layout Elemente zurückgreift.

\newpage
\section{Erste Schritte mit \LaTeX}
\LaTeX\, unterscheidet sich in der Arbeitsmodus recht stark von vielen anderen Dokumenten-erstellungs-anwendungen, die Sie vielleicht verwendet haben, wie z.B. Microsoft Word oder LibreOffice Writer: Diese Werkzeuge bieten den Benutzern eine interaktive Seite, in die sie ihren Text eingeben und bearbeiten können und verschiedene Formen von Formatierungen anwenden können.

\LaTeX\, funktioniert jedoch ganz anders: Stattdessen ist Ihr Dokument eine einfache Textdatei, die mit \LaTeX-Befehlen durchsetzt ist, die verwendet werden, um die gewünschten (gesetzten) Ergebnisse auszudrücken. Um ein sichtbares, gesetztes Dokument zu erstellen, wird Ihre \LaTeX-Datei von einer Software verarbeitet, die als \TeX-Engine bezeichnet wird, die die in Ihrer Textdatei eingebetteten Befehle verwendet, um den Satzprozess zu steuern und zu kontrollieren, wobei die \LaTeX-Befehle und der Dokumententext in eine professionell gesetzte PDF-Datei umgewandelt werden.

\subsection{Arbeitsweise mit VSCode}
Wenn Sie die nötigen Extensions in VSCode aktiviert haben, können Sie eine neue \textbf{.tex}-Datei im Code-Editor öffnen. Ist alles richtig eingestellt wie in dem Repository, dann sollte das neue PDF jeweils beim Speichern einer Datei gebaut werden. Oben rechts im Editor-Fenster finden Sie ein Icon, mit dem Sie das PDF auf der rechten Seite anzeigen können.

\vspace{3mm}
\begin{ex} Einstieg in \LaTeX

\begin{enumerate}
	\item Wie kann man im Code Editor Kommentare einfügen, ohne dass diese im PDF erscheinen?
	\item Versuchen Sie nachzuvollziehen, was die einzelnen Codezeilen bewirken. Schreiben Sie Ihre Erkenntnisse mit der Kommentarfunktion auf.
	\item Fügen Sie vor \verb|\begin{ex}| den Befehl \verb|\newpage| ein und kompilieren Sie Ihre Datei. Was beobachten Sie?
\end{enumerate}
\end{ex}
\vspace{3mm}

\noindent Sie werden merken, dass praktisch alle \LaTeX-Befehle mit einem ''\verb|\|'' \,beginnen. Zudem wird oft mit
\verb|\begin{...}| und \verb|\end{...}| gearbeitet. 
\vspace{1mm}\\

\noindent Unter \url{https://www.overleaf.com/learn/latex/Learn_LaTeX_in_30_minutes} finden Sie viele weitere Informationen zu Syntax und Aufbau.
 \newpage
\section{Grundlagen}
Einfache Textformatierung hilft dabei, wichtige Konzepte innerhalb eines Dokuments hervorzuheben und es lesbarer zu machen.

\subsection{Abstände}
''Unsichtbare'' Zeichen wie das Leerzeichen, Tabulatoren und das Zeilenende werden von \LaTeX\ einheitlich als Leerzeichen behandelt. \emph{Mehrere} Leerzeichen werden wie \emph{ein} Leerzeichen behandelt. Wenn man andere als die normalen Wort- und Zeilenabstände möchte, kann man dies also nicht durch die Eingabe von zusätzlichen Leerzeichen oder Leerzeilen erreichen, sondern nur mit entsprechenden \LaTeX\-Befehlen.\\
\emph{Eine} Leerzeile zwischen Textzeilen bedeutet das Ende eines Absatzes. \emph{Mehrere} Leerzeilen werden wie eine Leerzeile behandelt.

\vspace{3mm}
\noindent Sie haben die Möglichkeit, Abstände manuell zu steuern. Beispielsweise sorgt ''\verb|\\|'' für einen Zeilenumbruch, alternativ können Sie auch \verb|\newline| verwenden.\\
Für einen Seitenumbruch wird \verb|\newpage| oder auch \verb|\clearpage| verwendet.

\vspace{3mm}
\noindent {\bf Horizontale Abstände} können mit \verb|\hspace{...}| gesteuert werden, wobei Sie bei \verb|{...}| die Längenangabe inklusive Einheit definieren müssen. Soll der ganze Rest der Zeile leer gelassen werden, so können sie auch mit \verb|\hfill| arbeiten.

\noindent {\bf Vertikale Abstände} (zwischen Zeilen oder Absätzen) können mit \verb|\vspace{...}| gesteuert werden, wobei Sie bei \verb|{...}| die Längenangabe inklusive Einheit definieren müssen. Soll etwas zuunterst auf einer Seite geschrieben werden, so können Sie davor mit \verb|\vfill| arbeiten.

\vspace{3mm}
\noindent Wenn Sie vermeiden möchten, dass eine Zeile automatisch eingerückt wird, so können Sie \verb|\noindent| verwenden.

\newpage
\subsection{Textformatierungen}
Der Code \verb|\emph{Dies ist kursiv.}| erzeugt folgende Ausgabe: \emph{Dies ist kursiv}.

\vspace{2mm}
\noindent Andere häufige Anweisungen zur Textgestaltung sind:

\vspace{4mm}
\begin{tabular}{ll}
\verb|\textbf{argument}| & \textbf{Fetter Text} \\
\verb|\textit{argument}| &  \textit{kursiver Text} \\
\verb|\underline{argument}| &  \underline{unterstrichener Text} \\
\verb|{\tiny argument}| & \tiny{winziger Text} \\
\verb|{\small argument}| & {\small kleiner Text} \\ 
\verb|{\large argument}| & {\large grosser Text} \\
\verb|{\huge argument}| & {\huge riesiger Text}
\end{tabular}

\vspace{4mm}
\noindent Es ist auch möglich, Texte farbig zu gestalten. Dazu müssen Sie zunächst folgendes Package aufführen: \verb|\usepackage{xcolor}|. Dies können Sie beispielsweise auf Zeile 3 schreiben.\\
 Danach erzeugt der Code ''\verb|\textcolor{blue}{farbiger Text}|''  folgende Ausgabe: \textcolor{blue}{farbiger Text}

\vspace{4mm}
\begin{ex} \label{ex:1}
Ändern Sie zuerst in Overleaf den Abschnittstitel \emph{''Introduction''} zu \emph{''Aufgabe \ref{ex:1}''}.\\
Versuchen Sie im Anschluss den unten abgedruckten Text zu erzeugen:
\end{ex}

\vspace{2mm}
\noindent \textbf{Das ist ein formatierter Text.} \emph{Das ist ein formatierter Text.} \underline{Das ist ein formatierter Text.}\\
\textcolor{red}{Das ist ein formatierter Text.} \textcolor{green}{Das ist ein formatierter Text.} \textcolor{blue}{Das ist ein formatierter Text.}\\
\noindent {\tiny Das ist ein formatierter Text. Das ist ein formatierter Text. Das ist ein formatierter Text.}

\vspace{1cm}
\noindent {\huge Das ist ein formatierter Text. Das ist ein formatierter Text. Das ist ein formatierter Text.} \newpage
\section{Listen}
Es gibt drei Hauptarten von Listen, welche Sie nun kennenlernen werden: nummerierte Listen, Aufzählungslisten und beschreibende Listen.

\subsection{Nummerierte Listen}
Sie haben bereits weiter oben Nummerierte Listen gesehen. Die Aufgaben, welche Sie gelöst haben, wurden dadurch strukturiert und in kleinere Teile zerlegt.
Eine nummerierte Liste kann mit dem Befehl \verb|\begin{enumerate}| folgendermassen erzeugt werden:
\begin{verbatim}
\begin{enumerate}
\item Beginnen Sie jede nummerierte Liste mit einem \begin{enumerate}-Befehl.
\item Fügen Sie Ihre Listenelemente mit dem \item-Befehl hinzu.
\item Vergessen Sie nicht, Ihre Liste zu beenden.
\end{enumerate}
\end{verbatim}
Obiger Code wird im PDF-Dokument folgendermassen dargestellt:
\begin{enumerate}
\item Beginnen Sie jede nummerierte Liste mit einem \verb|\begin{enumerate}|-Befehl.
\item Fügen Sie Ihre Listenelemente mit dem \verb|\item|- Befehl hinzu.
\item Vergessen Sie nicht, Ihre Liste zu beenden.
\end{enumerate}


\subsection{Aufzählungslisten}
Manchmal möchten Sie jedoch eine Aufzählungsliste, auch bekannt als unnummerierte Liste. Aufzählungslisten funktionieren ähnlich wie nummerierte Listen.

\begin{verbatim}
\begin{itemize}
\item Beginnen Sie jede Aufzählungsliste mit einem \begin{itemize}-Befehl.
\item Fügen Sie Ihre Listenelemente mit dem \item-Befehl hinzu.
\item Vergessen Sie nicht, Ihre Liste zu beenden.
\end{itemize}
\end{verbatim}
Obiger Code wird im PDF-Dokument folgendermassen dargestellt:
\begin{itemize}
\item Beginnen Sie jede Aufzählungsliste mit einem \verb|\begin{itemize}|-Befehl.
\item Fügen Sie Ihre Listenelemente mit dem \verb|\item|-Befehl hinzu.
\item Vergessen Sie nicht, Ihre Liste zu beenden.
\end{itemize}


\subsection{Beschreibungslisten}
Eine weitere Möglichkeit bietet die Beschreibungsliste:
\begin{verbatim}
\begin{description}
\item[Aufzählungslisten] Nützlich für Listen von Materialien
\item[Nummerierte Listen] Nützlich für Verfahren
\item[Beschreibungslisten] Nützlich für Vokabeln
\end{description}
\end{verbatim}
Obiger Code wird im PDF-Dokument folgendermassen dargestellt:

\begin{description}
\item[Aufzählungslisten] Nützlich für Listen von Materialien
\item[Nummerierte Listen] Nützlich für Verfahren
\item[Beschreibungslisten] Nützlich für Vokabeln
\end{description}

\vspace{5mm}
\noindent Das spezielle an Beschreibungslisten ist, dass Sie ein Aufzählungssymbol Ihrer Wahl verwenden können:
\begin{verbatim}
\begin{description}
\item [$\star$] Ein Sternsymbol als Aufzählungssymbol.
\item [\#]  Ein Hashtag als Aufzählungssymbol.
\item [$\Rightarrow$] Ein Pfeil als Aufzählungssymbol.
\end{description}
\end{verbatim}
Obiger Code wird im PDF-Dokument folgendermassen dargestellt:
\begin{description}
\item [$\star$] Ein Sternsymbol als Aufzählungssymbol.
\item [\#]  Ein Hashtag als Aufzählungssymbol.
\item [$\Rightarrow$] Ein Pfeil als Aufzählungssymbol.
\end{description}

\vspace{3mm}
\begin{ex} \label{ex:4}
Fügen Sie wieder einen neuen Abschnittstitel \emph{''Aufgabe \ref{ex:4}''}  in Overleaf hinzu.\\
Erstellen Sie eine Liste mit vier Einträgen, welche als Aufzählungssymbol einen Kreis ($\circ$) verwendet. Tipp: \LaTeX\ -Cheat-Sheet
\end{ex} \newpage
\section{Abbildungen} 

Abbildungen werden mit der \verb|{figure}|-Umgebung und dem \verb|\includegraphics[ ]{}| Befehl erstellt, wobei die Grösse Ihrer Grafik in eckigen Klammern und der Dateiname in geschweiften Klammern platziert wird. Ein \verb|[h]| Zusatz nach dem \verb|\begin{figure}| Befehl teilt \LaTeX\ mit, das Bild an dieser bestimmten Stelle in Ihrem Dokument zu platzieren. Es wird auch empfohlen, dass Sie einen \verb|\label|- und \verb|\caption|- Befehl verwenden, um Ihre Abbildung ordnungsgemäss zu formatieren. Den Befehl \verb|\label{...}| werden Sie im nächsten Kapitel kennenlernen, damit kann man auf das gewünschte Bild verweisen. Eine ordnungsgemäss eingefügte Abbildung wird unten angezeigt.

\begin{figure}[h]
\centering 
\includegraphics[width=0.5\textwidth]{meme.jpg} 
\caption{Ein erstes Bild}
\label{fig:meme}
\end{figure}


\noindent Diese Grafik wurde mit dem folgenden Code eingefügt:
\begin{verbatim}
\begin{figure}[h]
\centering 
\includegraphics[width=0.5\textwidth]{meme.jpg} 
\caption{Ein erstes Bild}
\label{fig:meme}
\end{figure}
\end{verbatim}

\noindent Wenn Sie selbst ein Bild einfügen, so speichern Sie dieses am einfachsten links neben dem Code-Editor, dort wo sich auch Ihre Hauptdatei (\verb|main.tex|) befindet.\\ Zusätzlich können Sie die automatische englische Bezeichnung \emph{''figure''} zu \emph{''Abbildung''} ändern, indem Sie mit \verb|\usepackage[ngerman]{babel}| arbeiten.

\begin{ex}\label{ex:6}
Fügen Sie wieder einen neuen Abschnittstitel \emph{''Aufgabe \ref{ex:6}''}  in Overleaf hinzu.\\
Fügen Sie zwei Bilder Ihrer Wahl in verschiedenen Grössen ein.\\Verwenden Sie den \verb|\label{...}|-Befehl, um später damit arbeiten zu können.
\end{ex}

\newpage
\section{Querverweise und Hyperlinks}
Querverweise und Hyperlinks sind ein weiteres starkes Merkmal von \LaTeX\ . Das dazu benötigte \texttt{hyperref}-Paket ist bei Overleaf bereits aufgeführt, daher können Sie direkt damit arbeiten.\\
Querverweise werden gemacht, wenn Sie beispielsweise in einem Fliesstext auf einen anderen Abschnitt oder auf die Nummerierung einer Gleichung verweisen möchten.\\ Das praktische an der erzeugten PDF-Datei ist, dass der Querverweis ein Hyperlink ist, daher springt das Dokument nach Mausklick auf den Querverweis direkt zur richtigen Stelle.
\vspace{2mm}\\
Dafür müssen Sie zuerst die gewünschte Stelle, auf welche verwiesen werden soll, mit dem \verb|\label{...}|-Befehl kennzeichnen. Der Name, den Sie für den Verweis verwenden möchten, wird zwischen den geschweiften Klammern platziert. \\
Wenn Sie nun zu einem späteren Zeitpunkt auf die mit dem Label gekennzeichnete Stelle verweisen möchten, so können Sie dies mit dem \verb|\ref{labelname}|-Befehl tun, wobei labelname der Name ist, den Sie diesem Abschnitt zuvor gegeben haben.

\subsection{Links}
Hyperlinks oder Links zu Webadressen werden ebenfalls in \LaTeX\ unterstützt. Eine einfache URL kann in einem Dokument mit dem \verb|\url{}|-Befehl eingefügt werden, wobei die Webadresse einschliesslich des \texttt{http://}-Teils in die geschweiften Klammern eingesetzt wird. Der Weblink \url{https://gym-muttenz.ch/} wird mit dem Code \verb|\url{https://gym-muttenz.ch/}| erstellt. Damit dies Funktioniert, müssen sie noch oben in Ihrer Datei das \verb|\usepackage{url}| aufführen.\\Beachten Sie, dass die eingefügte URL in einem PDF-Dokument, das die Funktion unterstützt, ein aktiver Link ist.
\vspace{2mm}\\
\begin{ex} \label{ex:3}
Fügen Sie wieder einen neuen Abschnittstitel \emph{''Aufgabe \ref{ex:3}''}  in Overleaf hinzu.\\
\begin{enumerate}
	\item Schreiben Sie dann eine Zeile Text und fügen Sie einen Link Ihrer Wahl ein.
	\item Erstellen Sie einen Querverweis zu Ihrem Abschnitt  ''Aufgabe \ref{ex:6}''.
	\item Fügen Sie eine Zeile Text inklusive Querverweise zu Ihren obigen Bildern ein.
\end{enumerate}
\end{ex} \newpage
\section{Tabellen}
Tabellen werden mit der \verb|{tabular}|-Umgebung erstellt. Für beste Ergebnisse empfiehlt es sich jedoch, Ihre gesamte Tabelle innerhalb einer \verb|{table}|-Umgebung zu platzieren. Dadurch können Sie die Tabelle zentrieren, falls gewünscht, eine Beschriftung, ein Label hinzufügen und die Tabelle im Allgemeinen einfacher an anderer Stelle in Ihrem Bericht referenzieren. Spalten in Ihrer Tabelle werden durch das Kaufmannsund-Symbol ''\verb|&|'' getrennt. Wenn Sie einen vertikalen Separator zwischen Ihren Spalten möchten, platzieren Sie ein vertikales Trennsymbol ''\texttt{|}'' zwischen Ihren Ausrichtungszeichen.

\begin{table}[h]
\centering
\label{JustTable}
\begin{tabular}{|l|l|}
\hline
Code & Ausrichtungstyp \\
\hline
l & linksbündig \\
r & rechtsbündig \\
c & zentriert \\
\verb|\hline| & horizontale Trennlinie \\
$|$ & vertikaler Separator\\
\hline
\end{tabular}
\caption{Gemeinsame tabellarische Befehle}
\end{table}

\noindent Die obige Tabelle wurde mit folgendem Code erstellt.
\begin{verbatim}
\begin{table}[h]
\centering
\label{JustTable}
\begin{tabular}{|l|l|}
\hline
Code & Ausrichtungstyp \\
\hline
l & linksbündig \\
r & rechtsbündig \\
c & zentriert \\
\verb|\hline| & horizontale Trennlinie \\
$|$ & vertikaler Separator\\
\hline
\end{tabular}
\caption{Gemeinsame tabellarische Befehle}
\end{table}
\end{verbatim}


\newpage
\begin{ex}\label{ex:5}
Fügen Sie wieder einen neuen Abschnittstitel \emph{''Aufgabe \ref{ex:5}''}  in Overleaf hinzu.\\
Erstellen Sie die unten dargestellten Tabellen inklusive Text und Verweis.

\begin{table}[H]
\begin{center}
\begin{tabular}{|c|c|c|c|}
\hline
    Zeilennummer & Text & Formel & Zahl \\ \hline
    eins & Hallo & $a^2$ & 3  \\ \hline
    zwei & Fritzli & $2b-\sin(\alpha)$ & 13\\
\hline
\end{tabular}
\caption{Beispieltabelle}
\label{table:SW}
\end{center}
\end{table}


\begin{table}[H]
\begin{center}
\begin{tabular}{|cccc|}
\hline
\color{red}{Zeilennummer} &  \color{blue}{Text} & \color{green}{Formel} & Zahl \\ \hline \hline
    eins & Hallo & $a^2$ & 3  \\ \hline
    zwei & Fritzli & $2b-\sin(\alpha)$ & 13\\
\hline
\end{tabular}
\caption{Farbige Beispieltabelle}
\label{table:farbig}
\end{center}
\end{table}

\emph{Tabelle \ref{table:SW} ist schwarz-weiss, Tabelle \ref{table:farbig} ist farbig. }
\end{ex} \newpage
\section{Mathematik mit \LaTeX}
Der wahrscheinlich grösste Vorteil der Verwendung von \LaTeX\  liegt in der Fähigkeit, mathematische Gleichungen schnell und sauber zu notieren. Zum Beispiel ist die berühmte Gleichung $E = mc^2$ sehr einfach darzustellen, ebenso wie etwas Anspruchsvolleres wie $\sum_{n=1}^{\infty} 2^{-n}=1$.\\ Die Komplexität, die \LaTeX\ bewältigen kann, ist ziemlich erstaunlich. %Selbst Gleichungen mit unendlichen Summen wie $\sum_{n=1}^{\infty} 2^{-n}=1$ werden problemlos behandelt.

\subsection{Inline-Gleichungen}
Mathematische Ausdrücke, die sich im Fliesstext befinden, werden von Dollarzeichen (\verb|$|) umgeben. Eine einfache Gleichung wie $\vec{F}=m\cdot \vec{a}$ kann mit dem Code \verb|$\vec{F}=m\cdot \vec{a}$| inline platziert werden.

\subsection{Display-Gleichungen}
Mathematische Ausdrücke, die in einer neuen Zeile (und typischerweise zentriert) platziert werden sollen, werden als Display-Gleichungen bezeichnet und sind von den Symbolen \verb|\[...\]| umgeben. Die gleiche Formel wie oben im Display-Stil geschrieben würde folgendermassen aussehen:

\begin{verbatim}
\[
\vec{F}=m\cdot \vec{a}
\]
\end{verbatim}

\noindent und würde wie folgt aussehen:
\[
\vec{F}=m\cdot \vec{a}.
\] 

\noindent Wenn Sie Ihre Gleichungen nummerieren möchten, so können Sie auch mit \verb|\begin{equation}| und \verb|\end{equation}| arbeiten.

%\newpage
\begin{ex} \label{ex:2}
Fügen Sie zunächst einen weiteren Abschnittstitel \emph{''Aufgabe \ref{ex:2}''}  in Overleaf hinzu.\\
Versuchen Sie dann die unten dargestellten Texte und Formeln mit \LaTeX\ zu erzeugen.
Die Befehle für die mathematischen Symbole finden Sie im \LaTeX -Cheat-Sheet oder unter:\\
 \url{https://de.wikipedia.org/wiki/Hilfe:TeX}
\begin{enumerate}
	\item \emph{Für ein rechtwinkliges Dreieck gilt:} $c = \sqrt{a^2+b^2}$.
	\item \emph{Das Volumen einer Kugel mit Radius $r$ ist:} $\pi\int_{-r}^{r}r^2-x^2 \,dx$.
	\item \emph{Kleiner Gauß:} \[ 1+2+\cdot \cdot \cdot +n = \sum_{k=1}^{n}k=\frac{n(n+1)}{2}=\frac{n^2+n}{2} \]
	\item \emph{Eine Funktion} $f:X \rightarrow Y$ \emph{ist injektiv, falls}
		\begin{equation}
			f(x_1)=f(x_2) \Rightarrow x_1=x_2
		\end{equation}
\end{enumerate}
\end{ex} \newpage
\section{Der Aufbau einer Arbeit}

\subsection{Kapitel und Inhaltsverzeichnis}
Artikel werden in Abschnitte (\verb|\section{...}|) und Unterabschnitte (\verb|\subsection{...}|) unterteilt. \LaTeX\ nummeriert automatisch alle Abschnitte und Unterabschnitte.\\ Wenn Sie möchten, dass ein Abschnitt oder Unterabschnitt nicht nummeriert wird, fügen Sie ein Sternchen (\verb|*|) ein. 

\medskip
\noindent Beispielsweise wird \verb|\subsection{...}| nummerieret, \verb|\subsection*{...}| wird nicht nummeriert.

\medskip
\noindent Sollen alle Abschnitts- und Unterabschnitts-Titel in einem Inhaltsverzeichnis aufgeführt werden, so können Sie mit
\verb|\tableofcontents| arbeiten. Dort wo Sie diesen Befehl notieren, wird automatisch ein Inhaltsverzeichnis erzeugt.

\subsection{Quellen und Verzeichnisse}
Ein weiteres Argument für die Verwendung von \LaTeX\ ist die einfache Handhabung von Zitaten und Quellen. 
Man kann beispielsweise mit \texttt{biblatex} arbeiten, dabei wird eine externe Datei mit den Quellenangaben angelegt.
Dies kann folgendermassen erstellt werden:
\begin{enumerate}
	\item Erstellen Sie neben dem Code-Editor, dort wo auch Ihre Datei \texttt{main.tex} und die Bilder sind, eine neue Datei: \texttt{bibliography.bib}.
	\item Wenn Sie in dieser Datei ein ''@'' schreiben, so werden Ihnen einige Quellenarten vorgeschlagen.
	\item Wenn Sie sich für eine Quellenart entscheiden, so wird automatisch ein Angabenraster aufgeführt, in welches die Literatur-Informationen eingetragen werden können.
	\item Damit die Quellen genutzt werden können, müssen Sie in Ihrer \texttt{main.tex} Datei noch folgendes Package aufführen: 
\end{enumerate}
\qquad \qquad \verb|\usepackage[backend=biber,style=alphabetic,sorting=ynt]{biblatex} | 

\bigskip
\noindent Zudem benötigen Sie darunter \verb|\addbibresource{bibliography.bib}|, damit die Bibliographie-Datei importiert wird.
Mit \verb|\printbibliography[heading=bibintoc,title={...}]| wird das Quellenverzeichnis dann erzeugt.\\
Wenn man nun etwas zitieren möchten, kann man dies mit dem Befehl \verb|\cite{...}| tun. 

\bigskip
\noindent Schauen Sie sich unter folgendem Link noch ein Beispiel dazu an:\\
\url{https://www.overleaf.com/project/66150e42e998070cd671cfa4}

\newpage
\begin{ex}
Das Ziel dieser Aufgabe ist es, Ihr bisheriges Dokument als ''Arbeit'' zu gestalten.
\begin{enumerate}
	\item Fügen Sie mindestens drei neue Unterabschnitte hinzu.
	\item Erstellen Sie nach dem Titelblatt ein Inhaltsverzeichnis.
	\item Erstellen Sie eine Standardbibliographie (\texttt{bibliography.bib}) wie oben beschrieben.
	\item Erstellen Sie in Ihrer \texttt{bibliography.bib} Datei mindestens einen Bucheintrag und einen Artikeleintrag.
	\item Nutzen Sie in Ihrer \texttt{main.tex} Datei eine neue Seite, um das Quellenverzeichnis zu erstellen.
	\item Zitieren Sie Ihre Quellen an einem Ort Ihrer Wahl.
	\item Erstellen Sie auf der letzten Seite ein Abbildungsverzeichnis mit $\backslash$\texttt{listoffigures}.
	\end{enumerate}
\end{ex}
 \newpage

\end{document}