\section{Abbildungen} 

Abbildungen werden mit der \verb|{figure}|-Umgebung und dem \verb|\includegraphics[ ]{}| Befehl erstellt, wobei die Grösse Ihrer Grafik in eckigen Klammern und der Dateiname in geschweiften Klammern platziert wird. Ein \verb|[h]| Zusatz nach dem \verb|\begin{figure}| Befehl teilt \LaTeX\ mit, das Bild an dieser bestimmten Stelle in Ihrem Dokument zu platzieren. Es wird auch empfohlen, dass Sie einen \verb|\label|- und \verb|\caption|- Befehl verwenden, um Ihre Abbildung ordnungsgemäss zu formatieren. Den Befehl \verb|\label{...}| werden Sie im nächsten Kapitel kennenlernen, damit kann man auf das gewünschte Bild verweisen. Eine ordnungsgemäss eingefügte Abbildung wird unten angezeigt.

\begin{figure}[h]
\centering 
\includegraphics[width=0.5\textwidth]{meme.jpg} 
\caption{Ein erstes Bild}
\label{fig:meme}
\end{figure}


\noindent Diese Grafik wurde mit dem folgenden Code eingefügt:
\begin{verbatim}
\begin{figure}[h]
\centering 
\includegraphics[width=0.5\textwidth]{meme.jpg} 
\caption{Ein erstes Bild}
\label{fig:meme}
\end{figure}
\end{verbatim}

\noindent Wenn Sie selbst ein Bild einfügen, so speichern Sie dieses am einfachsten links neben dem Code-Editor, dort wo sich auch Ihre Hauptdatei (\verb|main.tex|) befindet.\\ Zusätzlich können Sie die automatische englische Bezeichnung \emph{''figure''} zu \emph{''Abbildung''} ändern, indem Sie mit \verb|\usepackage[ngerman]{babel}| arbeiten.

\begin{ex}\label{ex:6}
Fügen Sie wieder einen neuen Abschnittstitel \emph{''Aufgabe \ref{ex:6}''}  in Overleaf hinzu.\\
Fügen Sie zwei Bilder Ihrer Wahl in verschiedenen Grössen ein.\\Verwenden Sie den \verb|\label{...}|-Befehl, um später damit arbeiten zu können.
\end{ex}

\newpage
\section{Querverweise und Hyperlinks}
Querverweise und Hyperlinks sind ein weiteres starkes Merkmal von \LaTeX\ . Das dazu benötigte \texttt{hyperref}-Paket ist bei Overleaf bereits aufgeführt, daher können Sie direkt damit arbeiten.\\
Querverweise werden gemacht, wenn Sie beispielsweise in einem Fliesstext auf einen anderen Abschnitt oder auf die Nummerierung einer Gleichung verweisen möchten.\\ Das praktische an der erzeugten PDF-Datei ist, dass der Querverweis ein Hyperlink ist, daher springt das Dokument nach Mausklick auf den Querverweis direkt zur richtigen Stelle.
\vspace{2mm}\\
Dafür müssen Sie zuerst die gewünschte Stelle, auf welche verwiesen werden soll, mit dem \verb|\label{...}|-Befehl kennzeichnen. Der Name, den Sie für den Verweis verwenden möchten, wird zwischen den geschweiften Klammern platziert. \\
Wenn Sie nun zu einem späteren Zeitpunkt auf die mit dem Label gekennzeichnete Stelle verweisen möchten, so können Sie dies mit dem \verb|\ref{labelname}|-Befehl tun, wobei labelname der Name ist, den Sie diesem Abschnitt zuvor gegeben haben.

\subsection{Links}
Hyperlinks oder Links zu Webadressen werden ebenfalls in \LaTeX\ unterstützt. Eine einfache URL kann in einem Dokument mit dem \verb|\url{}|-Befehl eingefügt werden, wobei die Webadresse einschliesslich des \texttt{http://}-Teils in die geschweiften Klammern eingesetzt wird. Der Weblink \url{https://gym-muttenz.ch/} wird mit dem Code \verb|\url{https://gym-muttenz.ch/}| erstellt. Damit dies Funktioniert, müssen sie noch oben in Ihrer Datei das \verb|\usepackage{url}| aufführen.\\Beachten Sie, dass die eingefügte URL in einem PDF-Dokument, das die Funktion unterstützt, ein aktiver Link ist.
\vspace{2mm}\\
\begin{ex} \label{ex:3}
Fügen Sie wieder einen neuen Abschnittstitel \emph{''Aufgabe \ref{ex:3}''}  in Overleaf hinzu.\\
\begin{enumerate}
	\item Schreiben Sie dann eine Zeile Text und fügen Sie einen Link Ihrer Wahl ein.
	\item Erstellen Sie einen Querverweis zu Ihrem Abschnitt  ''Aufgabe \ref{ex:6}''.
	\item Fügen Sie eine Zeile Text inklusive Querverweise zu Ihren obigen Bildern ein.
\end{enumerate}
\end{ex}